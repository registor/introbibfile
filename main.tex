%\documentclass[fontset = none, t, aspectratio=169]{ctexbeamer}
\documentclass[fontset = none, t]{ctexbeamer}
% 载入需要的宏包
\input{setup/packages.tex}
% 进行必要的设置
\input{setup/format.tex}

% Information
\title[文献数据库]{\LARGE 参考文献数据库(*.bib)简单教程}
\author[N. Geng]{耿楠}
\institute[教发中心]{西北农林科技大学教学发展中心}
\date{\tosemester}
\titlegraphic{%
  \vspace{3.0cm}
  \qrcode[hyperlink, height=1.6cm]{https://github.com/registor/createbibfile}}

\begin{document}

\maketitle
\section[文献数据库]{参考文献数据库}
\subsection[概述]{概述}
\begin{frame}{文献数据库}{概述}
  \begin{spacing}{1.5}
    \begin{itemize}
    \item 文献生成基本原理
      \begin{itemize}
      \item \texinline{\thebibliography}环境结合\texinline{\bibitem}命
        令实现
        \begin{itemize}
        \item 直接写入tex源文件
        \item 外部程序生成
        \end{itemize}
      \end{itemize}
    \item 参考文献数据库
      \begin{itemize}
      \item 参考文献数据源
      \item 统一规范的\alert{文本文件}
      \item \enquote{\alert{bib}}为后缀名
      \end{itemize}
    \item 分类
      \begin{itemize}
      \item 传统Bib\TeX 格式,后端用Bibtex程序实现,4次编译
      \item 现代Bib\LaTeX 格式,后端用Biber程序实现,3次编译
      \end{itemize}
    \end{itemize}
  \end{spacing}
\end{frame}
\subsection[数据库构成]{数据库构成}
\begin{frame}[fragile]{文献数据库}{数据构成}
  \begin{spacing}{1.0}
    \begin{itemize}
    \item 参考文献数据库文件(\alert{*.bib}文件)
      \begin{itemize}
      \item 由参考文献记录构成的\alert{文本文件}\footnote[frame,2]{细
          节请使用texdoc bibtex和texdoc biblatex查阅相关文档。}
        \begin{itemize}
        \item 记录类型---Book(专著)、Article(期刊文章)等
        \item 记录域数据(数据项)---Title(题名)、Author(作者)等
        \end{itemize}
      \end{itemize}
    \end{itemize}
  \end{spacing}
  \vspace{-0.8ex}
  \begin{center}
    \begin{minipage}[h]{0.55\linewidth}
    \begin{textcb}{bib数据库文件格式}
      @Article{傅刚2000--,
        Title = {大风沙过后的思考},
        Author = {傅刚 and 赵承 and 李佳路},
        Date = {2000-04-12},
        Journaltitle = {北京青年报}
      }
      @Book{顾炎武1982--,
        Title = {昌平山水记},
        Author = {顾炎武},
        Publisher = {北京古籍出版社},
        Year = {1982},
        Location = {北京}
      }
    \end{textcb}
    \end{minipage}
  \end{center}
\end{frame}
\section[手动生成]{手动生成数据库文件}
% \section[软件工具]{文献数据库管理软件}
% \subsection[Zotero]{Zotero}
% \subsection[NoteExpress]{NoteExpress}
% \subsection[EndNote]{EndNote}
% \section[文献数据库切片]{文献数据库切片}
% \subsection[手动摘录]{手动摘录}
% \subsection[自动生成]{自动生成}

% \begin{frame}[standout,plain]
%   ...
% \end{frame}

\end{document}

%%% Local Variables:
%%% mode: latex
%%% TeX-master: t
%%% End:
