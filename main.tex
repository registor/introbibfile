%\documentclass[fontset = none, t, aspectratio=169]{ctexbeamer}
\documentclass[fontset = none, t]{ctexbeamer}
% 载入需要的宏包
\input{setup/packages.tex}
% 进行必要的设置
\input{setup/format.tex}

% Information
\title[文献数据库]{\LARGE 参考文献数据库(*.bib)简单教程}
\author[N. Geng]{耿楠}
\institute[教发中心]{西北农林科技大学教学发展中心}
\date{\tosemester}
\titlegraphic{%
  \vspace{3.0cm}
  \qrcode[hyperlink, height=1.6cm]{https://github.com/registor/createbibfile}}

\begin{document}

\maketitle
\section[文献数据库]{参考文献数据库}
\subsection[概述]{概述}
\begin{frame}{文献数据库}{概述}
  \begin{spacing}{1.5}
    \begin{itemize}
    \item 文献生成基本原理
      \begin{itemize}
      \item \texinline{\thebibliography}环境结合\texinline{\bibitem}命
        令实现
        \begin{itemize}
        \item 直接写入tex源文件
        \item 外部程序生成
        \end{itemize}
      \end{itemize}
    \item 参考文献数据库
      \begin{itemize}
      \item 参考文献数据源
      \item 统一规范的\alert{文本文件}
      \item \enquote{\alert{bib}}为后缀名
      \end{itemize}
    \item 分类
      \begin{itemize}
      \item 传统Bib\TeX 格式,后端用Bibtex程序实现,4次编译
      \item 现代Bib\LaTeX 格式,后端用Biber程序实现,3次编译
      \end{itemize}
    \end{itemize}
  \end{spacing}
\end{frame}
\subsection[数据库构成]{数据库构成}
\begin{frame}[fragile]{文献数据库}{数据构成}
  \begin{spacing}{1.0}
    \begin{itemize}
    \item 参考文献数据库文件(\alert{*.bib}文件)
      \begin{itemize}
      \item 由参考文献条目构成的\alert{文本文件}\footnote[frame,2]{细
          节请使用texdoc bibtex和texdoc biblatex查阅相关文档。}
        \begin{itemize}
        \item 条目类型---Book(专著)、Article(期刊文章)等
        \item 条目域数据(数据项)---Title(题名)、Author(作者)等
        \end{itemize}
      \end{itemize}
    \end{itemize}
  \end{spacing}
  \vspace{-2.0ex}
  \begin{center}
    \begin{minipage}[h]{0.55\linewidth}
    \begin{textcb}{bib数据库文件格式}
      @Article{傅刚2000--,
        Title = {大风沙过后的思考},
        Author = {傅刚 and 赵承 and 李佳路},
        Date = {2000-04-12},
        Journaltitle = {北京青年报}
      }
      @Book{顾炎武1982--,
        Title = {昌平山水记},
        Author = {顾炎武},
        Publisher = {北京古籍出版社},
        Year = {1982},
        Location = {北京}
      }
    \end{textcb}
    \end{minipage}
  \end{center}
\end{frame}
\section[手动生成]{手动生成数据库文件}
\begin{frame}[fragile]{手动生成}{使用记事本编辑}
  \begin{spacing}{1.8}
    \begin{itemize}
    \item notepad++、gedit、vim等文本编辑器\footnote[frame,2]{Windows记事本可能会有编码/
        换行等问题。}
      \begin{itemize}
      \item 需要符合Bib\TeX 格式或Bib\LaTeX 语法
      \item 根据实际情况填写不同条目及条目域数据
      \end{itemize}
    \end{itemize}    
    \begin{center}
      \includegraphics[height=0.42\textheight]{01bibnotepadpp}\quad
      \includegraphics[height=0.42\textheight]{01bibgedit}
    \end{center}
  \end{spacing}
\end{frame}

\begin{frame}[fragile]{手动生成}{使用TeXstudio编辑}
  \begin{spacing}{1.3}
    \begin{itemize}
    \item \menu{参考文献}菜单\footnote[frame,2]{可先用\menu{参考文献>
          类型}指定文献数据是Bib\TeX 或Bib\LaTeX 类型。}
      \begin{itemize}
      \item 插入指定类型条目
      \item 填写各条目域数据
      \item 可根据需要删减条目域
      \end{itemize}
    \end{itemize}
    \begin{center}
      \includegraphics[height=0.5\textheight]{02texstudiobibedit}
    \end{center}
  \end{spacing}  
\end{frame}

\begin{frame}[fragile]{手动生成}{嵌入\LaTeX 源文件}
  \begin{spacing}{1.3}
    \begin{itemize}
    \item 随\LaTeX 源文件生成文献数据库bib文件
      \begin{itemize}
      \item 需要\texinline{\usepackage{filecontents}}宏包
      \item 使用\texinline{filecontents}环境在导言区实现
      %\item 条目及条目域数据格式相同
      \item 编译后生成文献数据库bib文件
      \end{itemize}
    \end{itemize}
    \begin{center}
      \begin{minipage}[h]{0.55\linewidth}
        \begin{textcb}{随\LaTeX 源文件生成bib文件}
          \usepackage{filecontents}
          
          \begin{filecontents}{example.bib}
            @Article{傅刚2000--,
              Title = {大风沙过后的思考},
              Author = {傅刚 and 赵承 and 李佳路},
              Date = {2000-04-12},
              Journaltitle = {北京青年报}
            }
        \end{textcb}
      \end{minipage}
    \end{center}
  \end{spacing}  
\end{frame}

\begin{frame}[fragile]{手动生成}{使用JabRef编辑}
  \begin{spacing}{1.5}
    \begin{itemize}
    \item 免费开源*.bib数据库管理软件\link{http://www.jabref.org/}
      \begin{itemize}
      \item 生成*.bib数据库文件
      \item 添加文献数据条目
      \item 录入文献条目域数据
      \end{itemize}
    \end{itemize}    
    \begin{center}
      \includegraphics[width=0.48\textwidth]{03jabref01}\quad
      \includegraphics[width=0.48\textwidth]{03jabref02}
    \end{center}
  \end{spacing}
\end{frame}

\begin{frame}[fragile]{手动生成}{自动获取bib条目域数据}
  \begin{spacing}{1.5}
    \begin{itemize}
    \item 避免手动编写bib数据
      \begin{itemize}
      \item 百度学术(百度学术---保持学习的态
        度)\link{http://xueshu.baidu.com/}
      \item 单击检索结果条目下的\alert{\keys{引用}}按钮
      \item 单击\enquote{引用}中\enquote{导入链接}的\alert{\keys{BibTeX}}按钮
      \item 拷贝数据到bib数据库文件\footnote[frame,2]{必要时,再手动补全缺
          失域数据。}
      \end{itemize}
    \end{itemize}    
    \begin{center}
      \begin{annotatedFigure}
        {\includegraphics[width=0.48\textwidth]{04acdemicbaidu01}}
        \annotatedFigureBox{0.36,0.51}{0.46,0.58}{red}
        \annotatedFigureBox{0.44,0.015}{0.51,0.08}{blue}
      \end{annotatedFigure}\quad
      \begin{annotatedFigure}
        {\includegraphics[width=0.48\textwidth]{04acdemicbaidu02}}
        \annotatedFigureBox{0.005,0.05}{0.96,0.55}{blue}
      \end{annotatedFigure}
    \end{center}
  \end{spacing}
\end{frame}

\begin{frame}[fragile]{手动生成}{自动获取bib条目域数据}
  \begin{spacing}{1.5}
    \begin{itemize}
    \item 避免手动编写bib数据
      \begin{itemize}
      \item 必应学术(必应学术---学无止境,术有乾坤)\link{https://cn.bing.com/academic?mkt=zh-CN}
      \item 单击结果条目下的\alert{\keys{Cite}}按钮
      \item 单击\enquote{Cite}中\enquote{ImportLinks}的\alert{\keys{BibTeX}}按钮
      \item 拷贝数据到bib数据库文件\footnote[frame,2]{必要时,再手动补全缺
          失域数据。}
      \end{itemize}
    \end{itemize}    
    \begin{center}
      \begin{annotatedFigure}
        {\includegraphics[width=0.48\textwidth]{04acdemicbing01}}
        \annotatedFigureBox{0.165,0.34}{0.212,0.37}{red}
        \annotatedFigureBox{0.575,0.048}{0.622,0.082}{blue}
      \end{annotatedFigure}\quad
      \begin{annotatedFigure}
        {\includegraphics[width=0.48\textwidth]{04acdemicbing02}}
        \annotatedFigureBox{0.017,0.022}{0.995,0.73}{blue}
      \end{annotatedFigure}
    \end{center}
  \end{spacing}
\end{frame}

\begin{frame}[fragile]{手动生成}{自动获取bib条目域数据}
  \begin{spacing}{1.5}
    \begin{itemize}
    \item 避免手动编写bib数据
      \begin{itemize}
      \item Google学术
      \item 单击结果条目下的\alert{\keys{''}}按钮
      \item 单击\enquote{引用}中的\alert{\keys{BibTeX}}按钮
      \item 拷贝数据到bib数据库文件\footnote[frame,2]{必要时,再手动补全缺
          失域数据。}
      \end{itemize}
    \end{itemize}    
    \begin{center}
      \begin{annotatedFigure}
        {\includegraphics[width=0.48\textwidth]{04acdemicgoogle01}}
        \annotatedFigureBox{0.25,0.52}{0.285,0.565}{red}
        \annotatedFigureBox{0.635,0.037}{0.685,0.08}{blue}
      \end{annotatedFigure}\quad
      \begin{annotatedFigure}
        {\includegraphics[width=0.48\textwidth]{04acdemicgoogle02}}
        \annotatedFigureBox{0.031,0.022}{0.995,0.695}{blue}
      \end{annotatedFigure}
    \end{center}
  \end{spacing}
\end{frame}

\begin{frame}[fragile]{手动生成}{注意事项}
  \begin{spacing}{1.3}
    \begin{itemize}
    \item 特殊字符处理
      \begin{itemize}
      \item 对bib中的特殊字符需要使用命令替换
      \end{itemize}
    \end{itemize}
    \begin{center}
      \scriptsize
      \begin{tabular}[h]{|c|c|}        
        \hline
        字符 & 命令 \\ \hline
        \# & \texinline[fontsize=\scriptsize]{\#} \\ \hline
        \$ & \texinline[fontsize=\scriptsize]{\$} \\ \hline
        \% & \texinline[fontsize=\scriptsize]{\%} \\ \hline
        \& & \texinline[fontsize=\scriptsize]{\&} \\ \hline
        \_ & \texinline[fontsize=\scriptsize]{\_} \\ \hline
        \{ & \textbackslash\{ \\ \hline
        \} & \textbackslash\} \\ \hline
        \~{} & \texinline[fontsize=\scriptsize]{\~{}} \\ \hline
        \^{} & \texinline[fontsize=\scriptsize]{\^{}} \\ \hline
        $\backslash$ & \texinline[fontsize=\scriptsize]{$\backslash$} \\ \hline   
      \end{tabular}
    \end{center}
  \end{spacing}  
\end{frame}

\begin{frame}[fragile]{手动生成}{注意事项}
  \begin{spacing}{1.2}
    \begin{itemize}
    \item 责任者(author/editor)域数据的录入\footnote[frame,2]{参见胡振
        震开发的\enquote{符合 GB/T 7714-2015 标准的 biblatex 参考文献
          样式}宏包\link{https://github.com/hushidong/biblatex-gb7714-2015}}
      \begin{itemize}
      \item 姓名之间用\texinline{and}连接,省略用\texinline{others}
      \item 中文作者直接录入\\
        {\tiny 如:}\enquote{\texinline[fontsize=\tiny]{于潇 and 刘义 and 柴跃廷 and others}}
      \item 英文作者
        \begin{itemize}
        \item \texinline{prefix lastname, suffix, firstname middlename}
        \item \texinline{firstname middlename lastname}
        \item \texinline{firstname prefix lastname}
        \end{itemize}
        {\tiny 如:}\enquote{\texinline[fontsize=\tiny]{DES MARAIS, Jr., D J and H STRAUSS and SUMMONS, R. E. and others}}\\
        {\tiny 其中:第1个是\enquote{前缀,姓,后缀,名\ 中间名};第2个是\enquote{名\ 姓};第3个是\enquote{姓,名\ 中间名}。}\\
        {\tiny \alert{推荐}使用第1种方式,并且各组成部分首字母最好大写。}
      \item 中文机构名直接输入\\
        {\tiny 如:}\enquote{\texinline[fontsize=\tiny]{中国企业投资协会 and 台湾并购与私募股权协会 and 汇盈国际投资集团}}
      \item 英文机构名可能存在空格或and等字符串,最好用\enquote{\{\}}包围\\
        {\tiny 如:}\enquote{\texinline[fontsize=\tiny]{{International Federation of Library Association and Institutions} and NASA}}
      \end{itemize}
    \end{itemize}    
  \end{spacing}  
\end{frame}

\begin{frame}[standout,plain]
  太麻烦?\\
  试试文献管理软件...
\end{frame}
\section[文献管理软件]{文献管理软件}
\subsection[Zotero]{Zotero}
\begin{frame}[fragile]{文献管理软件}{Zotero}
  \begin{spacing}{1.5}
    \begin{itemize}
    \item 免费文献管理软件\link{https://www.zotero.org/}
      \begin{itemize}
      \item 安装浏览器插件和软件
      \item 收集文献条目\footnote[frame,2]{请参阅Zotero使用教程。}
      \item 导出文献数据库
      \end{itemize}
    \end{itemize}    
    \begin{center}
      \includegraphics[width=0.45\textwidth]{05zoterohomepage}
    \end{center}
  \end{spacing}
\end{frame}

\begin{frame}[fragile]{文献管理软件}{Zotero}
  \begin{spacing}{1.2}
    \begin{itemize}
    \item 以\alert{知网}文献检索为例
      \begin{itemize}
      \item 检索文献并选中文献
      \item 单击\keys{导出/参考文献}
      \end{itemize}
    \end{itemize}    
    \begin{center}
      \begin{annotatedFigure}
        {\includegraphics[width=0.65\textwidth]{05zoterocnki01}}
        \annotatedFigureBox{0.242,0.195}{0.275,0.515}{blue}
        \annotatedFigureBox{0.315,0.565}{0.415,0.595}{red}
      \end{annotatedFigure}      
    \end{center}
  \end{spacing}
\end{frame}

\begin{frame}[fragile]{文献管理软件}{Zotero}
  \begin{spacing}{1.2}
    \begin{itemize}
    \item 以\alert{知网}文献检索为例
      \begin{itemize}
      \item 选中导出的参考文献
      \item 单击地址栏后的\keys{Save to Zotero(CNKI)}按钮
      \end{itemize}
    \end{itemize}    
    \begin{center}
      \begin{annotatedFigure}
        {\includegraphics[width=0.9\textwidth]{05zoterocnki02}}
        \annotatedFigureBox{0.013,0.012}{0.046,0.37}{blue}
        \annotatedFigureBox{0.82,0.85}{0.855,0.91}{red}
      \end{annotatedFigure}      
    \end{center}
  \end{spacing}
\end{frame}

\begin{frame}[fragile]{文献管理软件}{Zotero}
  \begin{spacing}{1.2}
    \begin{itemize}
    \item 以\alert{知网}文献检索为例
      \begin{itemize}
      \item 通过\keys{Save to Zotero(CNKI)}按钮将文献条目导入Zotero文库
      \item 根据需要在Zotero文库中对条目数据进行修订
      \end{itemize}
    \end{itemize}    
    \begin{center}
      \includegraphics[width=0.49\textwidth]{05zoterocnki04}\\%\quad
      \includegraphics[width=0.49\textwidth]{05zoterocnki05}
    \end{center}
  \end{spacing}
\end{frame}

\begin{frame}[fragile]{文献管理软件}{Zotero}
  \begin{spacing}{1.2}
    \begin{itemize}
    \item 以\alert{知网}文献检索为例
      \begin{itemize}
      \item 选中需要导出的文库条目
      \item 右击选中的条目选择\menu{导出条目......}
      \item 选择指定的bib数据库文件格式导出为bib数据库文
        件\footnote[frame,2]{Zotero能够较为合理的处理\enquote{\&}等特
          殊字符。} \footnote[frame,3]{better
          BibLaTeX是Zotero的插件,用于生成更为合理的文件数据。}
      \end{itemize}
    \end{itemize}    
    \begin{center}
      \includegraphics[height=0.45\textheight]{05zoterocnki09}\quad
      \includegraphics[height=0.45\textheight]{05zoterocnki08}
    \end{center}
  \end{spacing}
\end{frame}

\begin{frame}[fragile]{文献管理软件}{Zotero}
  \begin{spacing}{1.2}
    \begin{itemize}
    \item 以\alert{知网}文献检索为例
      \begin{itemize}
      \item 根据需要对导出为bib数据库文件进行编辑
      \end{itemize}
    \end{itemize}    
    \begin{center}
      \includegraphics[width=0.95\textwidth]{05zoterocnki10}
    \end{center}
  \end{spacing}
\end{frame}
\subsection[NoteExpress]{NoteExpress}
\begin{frame}[fragile]{文献管理软件}{NoteExpress}
  \begin{spacing}{1.5}
    \begin{itemize}
    \item 中文文献管理软件\link{http://www.inoteexpress.com/aegean/}
      \begin{itemize}
      \item 收集文献条目\footnote[frame, 2]{请查阅NoteExpress使用手册
          及相关教程。}
      \item 导出Bib\TeX 格式文献数据库:\menu{文件>导出题录}
      \end{itemize}
    \end{itemize}    
    \begin{center}
      \includegraphics[width=0.7\textwidth]{06noteexpress01}
    \end{center}
  \end{spacing}
\end{frame}

\begin{frame}[fragile]{文献管理软件}{NoteExpress}
  \begin{spacing}{1.2}
    \begin{itemize}
    \item 导出题录
      \begin{itemize}
      \item 设置导出格式
        \begin{itemize}
        \item UTF-8
        \item Bib\TeX
        \end{itemize}
      \end{itemize}
    \end{itemize}
    \begin{center}
      \includegraphics[height=0.48\textheight]{06noteexpress02}
      \includegraphics[height=0.48\textheight]{06noteexpress03}
      %\includegraphics[height=0.45\textheight]{06noteexpress04}
    \end{center}
  \end{spacing}
\end{frame}

\begin{frame}[fragile]{文献管理软件}{NoteExpress}
  \begin{spacing}{1.2}
    \begin{itemize}
    \item 导出题录
      \begin{itemize}
      \item 更改文件后缀名为\enquote{.bib}
      \item 根据需要对导出为bib数据库文件进行编辑\footnote[frame,2]{可
          能需要处理\enquote{\&}等特殊字符、责任者名称中存在
          \enquote{空格或and}时需要分组等问题。}
      \end{itemize}
    \end{itemize}    
    \begin{center}
      \includegraphics[width=0.7\textwidth]{06noteexpress05}
    \end{center}
  \end{spacing}
\end{frame}
\subsection[EndNote]{EndNote}
\begin{frame}[fragile]{文献管理软件}{EndNote}
  \begin{spacing}{1.5}
    \begin{itemize}
    \item 经典文献管理软件\link{https://endnote.com/}
      \begin{itemize}
      \item 收集文献条目\footnote[frame, 2]{请查阅EndNote使用手册
          及相关教程。}
      \item 导出Bib\TeX 格式文献数据库:\menu{File>Export...}
      \end{itemize}
    \end{itemize}    
    \begin{center}
      \includegraphics[width=0.75\textwidth]{07endnote04}
    \end{center}
  \end{spacing}
\end{frame}

\begin{frame}[fragile]{文献管理软件}{EndNote}
  \begin{spacing}{1.2}
    \begin{itemize}
    \item 导出题录
      \begin{itemize}
      \item 设置导出格式
        \begin{itemize}
        \item Bib\TeX\ Export
        \end{itemize}
      \end{itemize}
    \end{itemize}
    \begin{center}
      \includegraphics[height=0.6\textheight]{07endnote05}\quad
      \includegraphics[height=0.6\textheight]{07endnote06}
    \end{center}
  \end{spacing}
\end{frame}

\begin{frame}[fragile]{文献管理软件}{EndNote}
  \begin{spacing}{1.2}
    \begin{itemize}
    \item 导出题录
      \begin{itemize}
      \item 更改文件后缀名为\enquote{.bib}
      \item 根据需要对导出为bib数据库文件进行编辑\footnote[frame,2]{可
          能需要处理\enquote{\&}等特殊字符、责任者名称中存在
          \enquote{空格或and}时需要分组等问题。}
      \end{itemize}
    \end{itemize}    
    \begin{center}
      \includegraphics[width=0.55\textwidth]{07endnote08}
    \end{center}
  \end{spacing}
\end{frame}

\begin{frame}[standout,plain]
  其它软件!\\
  哈,举一反三...
\end{frame}
\section[文献数据库切片]{文献数据库切片}
\begin{frame}[fragile]{文献数据库切片}{Biber程序}
  \begin{spacing}{1.5}
    \begin{itemize}
    \item 提取部分文献数据(切片)
      \begin{itemize}
      \item 大数据库$\Rightarrow$小数据库
      \item 仅适用于Bib\LaTeX 格式
      \item 需要用\LaTeX 编译后生成\enquote{jobname.bcf}文件
      \item 使用Biber程序处理\enquote{jobname.bcf}文件
      \end{itemize}
    \end{itemize}    
    \begin{center}
      \begin{minipage}[h]{0.9\textwidth}
        % \begin{textcb}{提取bib数据库切片命令}
        %   biber --output-format=bibtex jobname.bcf
        % \end{textcb}
        \begin{ubtdark}{xx@xx-lap:~}
          xx@xx-lap:~$ biber --output-format=bibtex jobname.bcf          
        \end{ubtdark}
      \end{minipage}
    \end{center}
  \end{spacing}
\end{frame}

\begin{frame}[standout,plain]
  感谢聆听!\\
  欢迎多提宝贵意见和建议
\end{frame}

\end{document}

%%% Local Variables:
%%% mode: latex
%%% TeX-master: t
%%% End:
